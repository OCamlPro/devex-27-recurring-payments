% !TeX spellcheck = en_US
\documentclass[10pt,a4paper]{article}
\usepackage[latin1]{inputenc}
\usepackage[T1]{fontenc}
\usepackage{amsmath}
\usepackage{amsfonts}
\usepackage{amssymb}
\usepackage{mathtools}
\usepackage{graphicx}
\usepackage{hyperref}
\hypersetup{
	colorlinks=true,
	linkcolor=blue,
	filecolor=magenta,
	urlcolor=gray
}

\usepackage{xcolor}
\usepackage{listings}

% Copyright 2017 Sergei Tikhomirov, MIT License
% https://github.com/s-tikhomirov/solidity-latex-highlighting/

\usepackage{listings, xcolor}

\definecolor{verylightgray}{rgb}{.97,.97,.97}

\lstdefinelanguage{Solidity}{
	keywords=[1]{anonymous, assembly, assert, balance, break, call, callcode, case, catch, class, constant, continue, constructor, contract, debugger, default, delegatecall, delete, do, else, emit, event, experimental, export, external, false, finally, for, function, gas, if, implements, import, in, indexed, instanceof, interface, internal, is, length, library, log0, log1, log2, log3, log4, memory, modifier, new, payable, pragma, private, protected, public, pure, push, require, return, returns, revert, selfdestruct, send, solidity, storage, struct, suicide, super, switch, then, this, throw, transfer, true, try, typeof, using, value, view, while, with, addmod, ecrecover, keccak256, mulmod, ripemd160, sha256, sha3}, % generic keywords including crypto operations
	keywordstyle=[1]\color{blue}\bfseries,
	keywords=[2]{address, bool, byte, bytes, bytes1, bytes2, bytes3, bytes4, bytes5, bytes6, bytes7, bytes8, bytes9, bytes10, bytes11, bytes12, bytes13, bytes14, bytes15, bytes16, bytes17, bytes18, bytes19, bytes20, bytes21, bytes22, bytes23, bytes24, bytes25, bytes26, bytes27, bytes28, bytes29, bytes30, bytes31, bytes32, enum, int, int8, int16, int24, int32, int40, int48, int56, int64, int72, int80, int88, int96, int104, int112, int120, int128, int136, int144, int152, int160, int168, int176, int184, int192, int200, int208, int216, int224, int232, int240, int248, int256, mapping, string, uint, uint8, uint16, uint24, uint32, uint40, uint48, uint56, uint64, uint72, uint80, uint88, uint96, uint104, uint112, uint120, uint128, uint136, uint144, uint152, uint160, uint168, uint176, uint184, uint192, uint200, uint208, uint216, uint224, uint232, uint240, uint248, uint256, var, void, ether, finney, szabo, wei, days, hours, minutes, seconds, weeks, years},	% types; money and time units
	keywordstyle=[2]\color{teal}\bfseries,
	keywords=[3]{block, blockhash, coinbase, difficulty, gaslimit, number, timestamp, msg, data, gas, sender, sig, value, now, tx, gasprice, origin},	% environment variables
	keywordstyle=[3]\color{violet}\bfseries,
	identifierstyle=\color{black},
	sensitive=false,
	comment=[l]{//},
	morecomment=[s]{/*}{*/},
	commentstyle=\color{gray}\ttfamily,
	stringstyle=\color{red}\ttfamily,
	morestring=[b]',
	morestring=[b]"
}

\lstset{
	language=Solidity,
	backgroundcolor=\color{verylightgray},
	extendedchars=true,
	basicstyle=\footnotesize\ttfamily,
	showstringspaces=false,
	showspaces=false,
	numbers=left,
	numberstyle=\footnotesize,
	numbersep=9pt,
	tabsize=2,
	breaklines=true,
	showtabs=false,
	captionpos=b
}


\lstdefinestyle{BashInputStyle}{
	language=bash,
	basicstyle=\small\sffamily,
	columns=fullflexible,
%	backgroundcolor=\color{yellow!20},
	linewidth=0.9\linewidth,
	xleftmargin=0.1\linewidth
}


\author{Fabrice Le Fessant\\OCamlPro SAS\\Telegram: @fabrice\_dune\\ \small{\url{https://github.com/OCamlPro/devex-27-recurring-payments}}}
\title{Recurring Payments Smart Contracts}
\date{\today}

\begin{document}
\maketitle

\section{Executive Summary}

This document describes a Recurring Payments System developed by
Fabrice LE FESSANT for FreeTON. The system allows {\em Service
  Providers} to publish {\em Services}, with a {\em cost per
  period}. Then, {\em Service Users} can subscribe to such services
for a predefined number of periods, paying for the whole
period. However, Service Providers can only spend the tokens for the
current period and already elapsed periods. Users can {\em stop} a
subscription (receiving back the tokens for the periods that will not
be consumed, or they can just {\em pause} the subscription (they can
later {\em unpause} such subscriptions).

The main property is that, to subscribe for a large set of periods,
only one transfer is needed. Another transfer will only happen if the
user decides to stop the subscription, and be refunded.

The System supports both TON tokens and TIP-3 tokens for all services
(a given Service must choose only one currency to use, but a Service
Provider can publish Services using different currencies). The current
System uses Broxus TIP-3 tokens.

Service users and service providers can monitor their contracts for
events, corresponding to changes in the subscriptions, or provide a
callback, i.e. a contract that will receive a message every time a
change is done on subscriptions.

The System has been deployed on the Testnet, and debots are available
for Service Providers and Service Users. A set of test scripts is also
available to test on TONOS-SE.

\tableofcontents

\section{Source code}

The source code of all smart contracts is available at
\url{https://github.com/OCamlPro/devex-27-recurring-payments}.

All smart contracts are written in Solidity, preprocessed with {\tt
  cpp -E} and {\tt ft} substitutions.

\section{Deployment}

The system has been deployed on the testnet.

The following addresses can be used:
\begin{description}
  \item[Root contract: 0:36b2657ba8546a1a7bdd06c9ce9062c7e998585a05fc00c03f57dc75546c96c4] The {\tt RPSRoot} contract shared by all
    contracts of a given System;
  \item[Provider debot: 0:af6e90a43565598f7b59a283416bacc2bebb222064c9b4d309b8e6725500fd6e] This debot can be used to act as a
    Service Provider, i.e. deploy a {\tt RPSProvider} contract, list
    services, add services and claim subscriptions for elapsed
    periods;
  \item[User debot: 0:ca8d11ffa44009a02ce4b02e42631e5c8c96042f1c0aae57acae132b29ed13a4] This debot can be used to act as a Service
    User, i.e deploy a {\tt RPSUser} contract, check subscriptions,
    subscribe to services and manage them (stop, pause and unpause);
  \item[Provider contract: 0:bd6c0225544eac8f4314e33d88b7e252cf6d5e002cf77654eeb4a03e11c5ec7f] This is an example Service Provider
    contract with only one service, with a period of 1 minute, and a
    cost of 1 ton per minute;
    
\end{description}

These contracts have been deployed with TVC images corresponding to
GIT commit.

\section{Architecture}

\subsection{Smart contracts}

The System provides 3 different kinds of contracts:
\begin{description}
\item[{\tt RPSRoot} contract:] the {\tt RPSRoot} contract is in charge
  of deploying smart contracts for Service Providers and Service
  Users.

  The code is available here: \url{https://github.com/OCamlPro/devex-27-recurring-payments/blob/master/contracts/RPSRoot.spp}

  The contrat is used as follows:
  \begin{itemize}
  \item After deployment, the owner must call functions {\tt
    setProviderCode} and {\tt setUserCode} to provide the code of
    other contracts.
  \item Before deploying a contract, a user must first provide some
    funds to the contract. This is done by calling the {\tt
      creditBalance} function, to associate the funds with the user's
    pubkey. Once his balance contains enough TONs, the user can either
    call {\tt deployProvider} or {\tt deployUser} functions.
  \item The contract also provides utility functions, such as {\tt
    getUserAddress} and {\tt getProviderAddress} to compute addresses
    from pubkeys.
  \end{itemize}
  
\item[{\tt RPSProvider} contract:] the {\tt RPSProvider} contract is
  in charge of managing the services and subscriptions for a given
  Service Provider (identified either by his pubkey or his address).

  The code is available here: \url{https://github.com/OCamlPro/devex-27-recurring-payments/blob/master/contracts/RPSProvider.spp}

  This contract is used as follows:
  \begin{itemize}
  \item To manage Services, the contract provides two functions {\tt
    addService} and {\tt getServices}. Services have a {\tt name}, a
    {\tt description}, a {\tt period} (in seconds) and a {\tt
      period\_cost} (in nanotons). Once added, services receive a uniq
    identifier for the provider, called {\tt serv\_id}.
  \item The contract provides different entry points that can only be
    called by {\tt RPSUser} contracts. They are used to get
    information on services, subscribe to services, and then stop,
    pause and unpause subscriptions.
  \item Finally, the Service Provider can claim the tokens that can
    already be spent: they correspond to elapsed periods and already
    started periods. Tokens for periods in the future are locked and
    cannot be used, so that the user can be refunded if he stops the
    subscription. To claim tokens, the Service Provider must first
    call {\tt claimSubscriptions} to unlock tokens for all
    subscriptions. Once tokens have been unlocked, the Service
    Provider can call the {\tt transferClaimed} function, providing
    the token root (address 0 is used for TONs) and the destination
    wallet.
  \end{itemize}
  
\item[{\tt RPSUser} contract:] the {\tt RPSUser} contract is in charge
  of managing subscriptions for a given user (identified either by a
  pubkey or an address).
  
  The code is available here: \url{https://github.com/OCamlPro/devex-27-recurring-payments/blob/master/contracts/RPSUser.spp}

  The contract is used as follows:
  \begin{itemize}
  \item The user must transfer tokens to the contract to be able to
    subscribe to services. For that, the user may use {\tt needWallet}
    to create a wallet for a specific TIP-3 token, and {\tt getWallet}
    to get the address of the wallet, and its current balance. He can
    then transfer tokens to the wallet. He can also use {\tt transfer}
    to recover some of the tokens to his personal wallet.
  \item The user can use {\tt subscribe} to subscribe to a particular
    service. He must provide {\tt provider}, the address of the
    Service Provider contract, {\tt serv\_id}, the identifier of the
    service, {\tt periods}, the number of periods, and {\tt callback},
    the address of a callback contract (or the address 0). The user
    can use {\tt getSubscriptions} to check if the service was
    correctly added. Events and callbacks are used to warn the user if
    a problem occurred and the service could not be subscribed.
  \item For every subscription, the user can call {\tt stopSubscribe},
    {\tt pauseSubscribe} and {\tt unpauseSubscribe}. {\tt
      stopSubscribe} will refund the user for periods that have not
    yet started. {\tt pauseSubscribe} pauses the subscription, and
    {\tt unpause} continues a paused subscription, with exactly the
    same remaining time as when it was paused.
  \end{itemize}
  
\end{description}

\subsection{Debots}

The System provides 2 different kinds of debots:

\begin{description}
\item[{\tt RPSUserDebot}:] This debot allows a user to manage his
  subscriptions. It must be initialized with the address of the root
  contract. If the {\tt RPSUser} contract of the user has not yet been
  deployed, the debot will transfer an initial credit from the user's
  multisig to the root contract, and then deploy the contract.

  The code is available here: \url{https://github.com/OCamlPro/devex-27-recurring-payments/blob/master/debot/RPSUserDebot.spp}

  
\item[{\tt RPSProviderDebot}:] This debot allows a Service Provider to
  manage his services and claim his subscriptions. It must be
  initialized with the address of the root contract. If the {\tt
    RPSProvider} contract has not yet been deployed, the debot will
  transfer an initial credit from the owner's multisig to the root
  contract, and then deploy the contract.

  The code is available here: \url{https://github.com/OCamlPro/devex-27-recurring-payments/blob/master/debot/RPSProviderDebot.spp}
  
\end{description}

\subsection{Multiple Currencies}

Both {\tt RPSUser} and {\tt RPSProvider} inherits from {\tt
  MultiWallet} contract, a contract that is used to manage multiple
wallets for multiple TIP-3 tokens. The address 0 is reserved for TONs.

  The code is available here: \url{https://github.com/OCamlPro/devex-27-recurring-payments/blob/master/contract/MultiWallet.spp}

\subsection{Development Tools}

We provide debots so that the contracts can be easily tested. For the
development itself, we used {\tt ft},
\url{https://github.com/OCamlPro/freeton_wallet}, both to build
Solidity contracts, deploy them and test them.

During the contest, many new features were added to {\tt ft},
especially to ease the development of debots. In particular, Solidity
files with {\tt spp} extension are preprocessed using {\tt cpp -E},
and we provide a set of macros to simplify the development of depots.
\url{https://github.com/OCamlPro/devex-27-recurring-payments/blob/master/debot/lib/cpp.sol}


\end{document}
