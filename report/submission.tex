% !TeX spellcheck = en_US
\documentclass[10pt,a4paper]{article}
\usepackage[latin1]{inputenc}
\usepackage[T1]{fontenc}
\usepackage{amsmath}
\usepackage{amsfonts}
\usepackage{amssymb}
\usepackage{mathtools}
\usepackage{graphicx}
\usepackage{hyperref}
\hypersetup{
	colorlinks=true,
	linkcolor=blue,
	filecolor=magenta,
	urlcolor=gray
}

\usepackage{xcolor}
\usepackage{listings}

% Copyright 2017 Sergei Tikhomirov, MIT License
% https://github.com/s-tikhomirov/solidity-latex-highlighting/

\usepackage{listings, xcolor}

\definecolor{verylightgray}{rgb}{.97,.97,.97}

\lstdefinelanguage{Solidity}{
	keywords=[1]{anonymous, assembly, assert, balance, break, call, callcode, case, catch, class, constant, continue, constructor, contract, debugger, default, delegatecall, delete, do, else, emit, event, experimental, export, external, false, finally, for, function, gas, if, implements, import, in, indexed, instanceof, interface, internal, is, length, library, log0, log1, log2, log3, log4, memory, modifier, new, payable, pragma, private, protected, public, pure, push, require, return, returns, revert, selfdestruct, send, solidity, storage, struct, suicide, super, switch, then, this, throw, transfer, true, try, typeof, using, value, view, while, with, addmod, ecrecover, keccak256, mulmod, ripemd160, sha256, sha3}, % generic keywords including crypto operations
	keywordstyle=[1]\color{blue}\bfseries,
	keywords=[2]{address, bool, byte, bytes, bytes1, bytes2, bytes3, bytes4, bytes5, bytes6, bytes7, bytes8, bytes9, bytes10, bytes11, bytes12, bytes13, bytes14, bytes15, bytes16, bytes17, bytes18, bytes19, bytes20, bytes21, bytes22, bytes23, bytes24, bytes25, bytes26, bytes27, bytes28, bytes29, bytes30, bytes31, bytes32, enum, int, int8, int16, int24, int32, int40, int48, int56, int64, int72, int80, int88, int96, int104, int112, int120, int128, int136, int144, int152, int160, int168, int176, int184, int192, int200, int208, int216, int224, int232, int240, int248, int256, mapping, string, uint, uint8, uint16, uint24, uint32, uint40, uint48, uint56, uint64, uint72, uint80, uint88, uint96, uint104, uint112, uint120, uint128, uint136, uint144, uint152, uint160, uint168, uint176, uint184, uint192, uint200, uint208, uint216, uint224, uint232, uint240, uint248, uint256, var, void, ether, finney, szabo, wei, days, hours, minutes, seconds, weeks, years},	% types; money and time units
	keywordstyle=[2]\color{teal}\bfseries,
	keywords=[3]{block, blockhash, coinbase, difficulty, gaslimit, number, timestamp, msg, data, gas, sender, sig, value, now, tx, gasprice, origin},	% environment variables
	keywordstyle=[3]\color{violet}\bfseries,
	identifierstyle=\color{black},
	sensitive=false,
	comment=[l]{//},
	morecomment=[s]{/*}{*/},
	commentstyle=\color{gray}\ttfamily,
	stringstyle=\color{red}\ttfamily,
	morestring=[b]',
	morestring=[b]"
}

\lstset{
	language=Solidity,
	backgroundcolor=\color{verylightgray},
	extendedchars=true,
	basicstyle=\footnotesize\ttfamily,
	showstringspaces=false,
	showspaces=false,
	numbers=left,
	numberstyle=\footnotesize,
	numbersep=9pt,
	tabsize=2,
	breaklines=true,
	showtabs=false,
	captionpos=b
}


\lstdefinestyle{BashInputStyle}{
	language=bash,
	basicstyle=\small\sffamily,
	columns=fullflexible,
%	backgroundcolor=\color{yellow!20},
	linewidth=0.9\linewidth,
	xleftmargin=0.1\linewidth
}


\author{Fabrice Le Fessant\\OCamlPro SAS\\Telegram: @fabrice\_dune\\ \small{\url{https://github.com/OCamlPro/devex-27-recurring-payments}}}
\title{Recurring Payments Smart Contracts}
\date{\today}

\begin{document}
\maketitle

\section{Executive Summary}

This document describes a Recurring Payments System developed by
Fabrice LE FESSANT for FreeTON. The system allows {\em Service
  Providers} to publish {\em Services}, with a {\em cost per
  period}. Then, {\em Service Users} can subscribe to such services
for a predefined number of periods, paying for the whole
period. However, Service Providers can only spend the tokens for the
current period and already elapsed periods. Users can {\em stop} a
subscription (receiving back the tokens for the periods that will not
be consumed, or they can just {\em pause} the subscription (they can
later {\em unpause} such subscriptions).

The main property is that, to subscribe for a large set of periods,
only one transfer is needed. Another transfer will only happen if the
user decides to stop the subscription, and be refunded.

The System supports both TON tokens and TIP-3 tokens for all services
(a given Service must choose only one currency to use, but a Service
Provider can publish Services using different currencies). The current
System uses Broxus TIP-3 tokens.

The System has been deployed on the Testnet, and debots are available
for Service Providers and Service Users. A set of test scripts is also available to test on TONOS-SE.

\tableofcontents

\section{Source code}

The source code of all smart contracts is available at
\url{https://github.com/OCamlPro/devex-27-recurring-payments}.

\section{Deployment}

The system has been deployed on the testnet.

The following addresses can be used:
\begin{description}
  \item[Root contract: XXX] The {\tt RPSRoot} contract shared by all
    contracts of a given System;
  \item[Provider debot: XXX] This debot can be used to act as a
    Service Provider, i.e. deploy a {\tt RPSProvider} contract, list
    services, add services and claim subscriptions for elapsed
    periods;
  \item[User debot: XXX] This debot can be used to act as a Service
    User, i.e deploy a {\tt RPSUser} contract, check subscriptions,
    subscribe to services and manage them (stop, pause and unpause);
  \item[Provider contract: XXX] This is an example Service Provider
    contract with only one service, with a period of 1 minute, and a
    cost of 1 ton per minute;
    
\end{description}
    
\section{Architecture}

\subsection{Smart contracts}

The System provides 3 different kinds of contracts:
\begin{description}
\item[{\tt RPSRoot} contract]: the {\tt RPSRoot} contract is in charge
  of deploying smart contracts for Service Providers and Service
  Users.
\item[{\tt RPSUser} contract]
\item[{\tt RPSProvider} contract]
\end{description}

\subsection{Debots}

The System provides 2 different kinds of debots:

\begin{description}
\item[{\tt RPSUserDebot}]
\item[{\tt RPSProviderDebot}]
\end{description}

\subsection{Multiple Currencies}

\subsection{Development Tools}

\end{document}
